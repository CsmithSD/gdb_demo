 \documentclass[11pt]{beamer}
\usetheme{Warsaw}
\usepackage[utf8]{inputenc}
\usepackage{amsmath}
\usepackage{amsfonts}
\usepackage{amssymb}
\usepackage{tikz}
\usepackage{xcolor}
\usepackage{listings}
\usepackage{caption}
\usepackage{hyperref}
\hypersetup{
    colorlinks=true,
    linkcolor=blue,
    filecolor=magenta,      
    urlcolor=cyan,
}
\definecolor{ceil}{rgb}{0.57, 0.63, 0.81}
\lstdefinestyle{BashInputStyle}{
  language=bash,
  basicstyle=\small\sffamily,
  %numbers=left,
  %numberstyle=\tiny,
  %numbersep=3pt,
  frame=tb,
  columns=fullflexible,
  backgroundcolor=\color{ceil!20},
  linewidth=0.9\linewidth,
  xleftmargin=0.1\linewidth
}


 
\urlstyle{same}
\usetikzlibrary{arrows,automata}
%\usepackage[latin1]{inputenc}

\author{Chris Smith}
\title{GDB Information}
%\setbeamercovered{transparent} 
%\setbeamertemplate{navigation symbols}{} 
%\logo{} 
\institute{SDSMT ACM/LUG} 
%\date{} 
%\subject{} 
\begin{document}

\begin{frame}
\titlepage
\end{frame}

\begin{frame}{Overview}
\begin{itemize}
\item What is gdb?
\item Compiling with debugging Flags
\item Running gdb
\item Interactive Shell
\item Running the Program
\item Breakpoints
\item Stepping Through
\item Viewing Variables
\item More Commands
\item Conditional Breakpoints
\end{itemize}
\end{frame}


\begin{frame}{What is gdb?}
\begin{itemize}
\item The GNU Debugger
\item Debugs C/C++, Go, Objective-C, and more
\item Allows for viewing what the program is executing
\item \href{https://www.gnu.org/software/gdb/documentation/}{www.gnu.org/softwoare/gdb/documentation}
\end{itemize}
\end{frame}

\begin{frame}[fragile]
\frametitle{Compiling}
Typically programs are compiled by running:
\begin{lstlisting}[style=BashInputStyle]
 $ g++ [flags] prog1.cpp -o prog1
\end{lstlisting}
To enable debugging flags which gdb needs, add the -g flag like so:
\begin{lstlisting}[style=BashInputStyle]
 $ g++ [flags] -g prog1.cpp -o prog1
\end{lstlisting}
\end{frame}

\begin{frame}[fragile]
\frametitle{Running gdb}
To start gdb run the following:
\begin{lstlisting}[style=BashInputStyle]
 $ gdb
\end{lstlisting}
 To load a file on start up just add the file name as a commandline argument. \\
 After the above command the following prompt should be shown:
\begin{lstlisting}[style=BashInputStyle]
 (gdb)
\end{lstlisting}
\end{frame}

\begin{frame}{Interactive Shell}
gdb is like the interactive shells you get when first logging into Linux.
\begin{itemize}
\item Similar to bash, python, and lisp
\item History of commands
\item Tab completion
\item Execution of shell commands
\end{itemize}
If there ever is any confusion the built in help menu gives a description of what a command does and what arguments it takes.
\end{frame}

\begin{frame}[fragile]
\frametitle{Running the program}
To run the the program:
\begin{lstlisting}[style=BashInputStyle]
(gdb) run
\end{lstlisting}
If there are no serious problems it should execute normally.
\begin{lstlisting}[style=BashInputStyle]
[inferior 1 (process 1234) exited normally]
\end{lstlisting}
If there are is a problem it will return information regarding where it crashed, what function, and the function variables that caused the error.
\begin{lstlisting}[style=BashInputStyle]
Program received signal SIGSEGV, Segmentation fault.
0x0000000000400633 in main () at fib.cpp:27
27        x[c] = next;
\end{lstlisting}
\end{frame}

\begin{frame}{Debugging}
The program is running successfully without any segmentation faults. However, the result isn't what is expected.
\break
\break
Downside to using console output:
\begin{itemize}
\item Cout, cin, printf, and scanf are slow
\item Information can scroll by too fast
\item Memory allocation
\item May not be thread safe
\item Hinders timing sensitive programs
\end{itemize}
\end{frame}
\begin{frame}{Breakpoints}
Stepping through the code can be a more efficient approach since a segmentation fault could be happening in one function but the root of the cause was a couple of function calls ago.
\break
\break
Breakpoints in gdb are similar to what is used in Visual Studio. They allow for you to stop the program from executing once it reaches a designated spot in the code.
\end{frame}

\begin{frame}[fragile]
\frametitle{Breakpoints}
To use a break just enter the command \textbf{\textcolor{blue}{break}} followed by the file and line number pair like so:
\begin{lstlisting}[style=BashInputStyle]
(gdb) break fib.cpp:27
\end{lstlisting}
This will set a break point in file fib.cpp at line 27. Anytime this line is reached the program will stop executing until another command is entered. 
\end{frame}

\begin{frame}[fragile]
\frametitle{Breakpoints}
Breakpoints can also be used to break at functions instead of line numbers. If there existed a function called sum\_array in fib.cpp the following could be done:
\begin{lstlisting}[style=BashInputStyle]
(gdb) break sum_array
\end{lstlisting}
gdb will stop executing once this function is reached.
\end{frame}

\begin{frame}{Stepping Through}
Now that the breakpoints are set we can use the \textbf{\textcolor{blue}{run}} command again to start executing the code. It will stop at these breakpoints unless an error occurs before reaching this point.
\break
\break
Once the point is reached you can proceed to the next breakpoint by using the command \textbf{\textcolor{blue}{continue}}. The command run would restart the execution from the beginning.
\break
\break
Instead of continuing on to the next breakpoint one can go to the next line of code by using the command \textbf{\textcolor{blue}{step}}. 
\end{frame}
\begin{frame}{Stepping Through}
Be careful with step because it will go into the sub-routine like printf or scanf. If you don't want to go into the sub-routine use the command \textbf{\textcolor{blue}{next}} and it will treat it as a instruction.
\break
\break
Retyping the same command over and over can get tedious so gdb allows the user to just hit enter and will execute the previously entered command.
\end{frame}

\begin{frame}{Viewing Variables}
Breakpoints alone are not useful when variables are involved. They allow you to see if a function or if the interior of a conditional statement is being executed, however, they don't allow the user to view the value of a variable.
\end{frame}

\begin{frame}[fragile]
\frametitle{Viewing Variables}
If you want to see all the variables in the local function that the program has stopped executing in due to a breakpoint type the command \textbf{\textcolor{blue}{info}} followed by the scope that is to be referenced. \\
For example to view local variables type the following:
\begin{lstlisting}[style=BashInputStyle]
(gdb) info locals
\end{lstlisting}
To view arguments or global variables replace locals in the above command with args and variables respectfully.
\end{frame}

\begin{frame}[fragile]
\frametitle{Viewing Variables}
In order to view one variable the \textbf{\textcolor{blue}{print}} command followed by a variable name will print out the value of the variable. If hex output is desired use \textbf{\textcolor{blue}{print/x}}
\begin{lstlisting}[style=BashInputStyle]
(gdb) print my_var
(gdb) print/x my_var
\end{lstlisting}
\end{frame}

\begin{frame}{More Commands}
Useful Commands:
\begin{itemize}
\item \textbf{\textcolor{blue}{backtrace}} - summary of how your program got where it is (Done when a seg fault occurs)
\item \textbf{\textcolor{blue}{where}} - gives a function call trace of how you got to this point and shows line numbers inside files (Can be done anywhere)
\item \textbf{\textcolor{blue}{finnish}} - runs until the current function has completed executing
\item \textbf{\textcolor{blue}{delete}} - deletes a specific breakpoint
\item \textbf{\textcolor{blue}{info breakpoints}} - shows information about all breakpoints
\end{itemize}
Another useful way to run gdb is to either use gdb --tui to bring up the source file and the interactive shell. If a gui is desired for gdb you can run ddd followed by the program executable. 
\end{frame}


\begin{frame}{Conditional Breakpoints}
Breakpoints are useful to figure out what the problem is or the general issue with a variable. 
\break
\break
We don't want to keep stepping through a function if the issue only happens towards the end of an array. 
\break
\break
So for this reason we would like to stop executing based on a conditional requirement.
\end{frame}

\begin{frame}[fragile]
\frametitle{Conditional Breakpoints}
Conditional breakpoints are just like normal breakpoints except for the extra conditional:
\begin{lstlisting}[style=BashInputStyle]
(gdb) break fib.cpp:27 if c >= SIZE
(gdb) break fib.cpp:27 if(c >= SIZE || c < 0)
\end{lstlisting}
\end{frame}
\end{document}









