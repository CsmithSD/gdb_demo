\documentclass[11pt]{beamer}
\usetheme{Warsaw}
\usepackage[utf8]{inputenc}
\usepackage{amsmath}
\usepackage{amsfonts}
\usepackage{amssymb}
\usepackage{tikz}
\usepackage{xcolor}
\usepackage{listings}

\usepackage{hyperref}
\hypersetup{
    colorlinks=true,
    linkcolor=blue,
    filecolor=magenta,      
    urlcolor=cyan,
}
\definecolor{ceil}{rgb}{0.57, 0.63, 0.81}
\lstdefinestyle{BashInputStyle}{
  language=bash,
  basicstyle=\small\sffamily,
  %numbers=left,
  %numberstyle=\tiny,
  %numbersep=3pt,
  frame=tb,
  columns=fullflexible,
  backgroundcolor=\color{ceil!20},
  linewidth=0.9\linewidth,
  xleftmargin=0.1\linewidth
}

\urlstyle{same}
\usetikzlibrary{arrows,automata}
%\usepackage[latin1]{inputenc}

\author{Chris Smith}
\title{GDB Information}
%\setbeamercovered{transparent} 
%\setbeamertemplate{navigation symbols}{} 
%\logo{} 
\institute{SDSMT ACM/LUG} 


\begin{document}

\begin{frame}
\titlepage
\end{frame}

\begin{frame}{Overview}
\begin{itemize}
\item What is gdb?
\item Compiling with debugging Flags
\item Running gdb
\item Interactive Shell
\item Running the Program
\item Breakpoints
\item Stepping Through
\item Viewing Variables
\item More Commands
\item Conditional Breakpoints
\end{itemize}
\end{frame}


\begin{frame}{What is gdb?}
\begin{itemize}
\item The GNU Debugger
\item Debugs C/C++, Go, Objective-C, and more
\item Allows for viewing what the program is executing
\item \href{https://www.gnu.org/software/gdb/documentation/}{www.gnu.org/softwoare/gdb/documentation}
\end{itemize}
\end{frame}

\begin{frame}[fragile]
\frametitle{Compiling}
Typically programs are compiled by running:
\begin{lstlisting}[style=BashInputStyle]
 $ g++ [flags] prog1.cpp -o prog1
\end{lstlisting}
To enable debugging flags which gdb needs, add the -g flag like so:
\begin{lstlisting}[style=BashInputStyle]
 $ g++ [flags] -g prog1.cpp -o prog1
\end{lstlisting}
\end{frame}

\begin{frame}[fragile]
\frametitle{Running gdb}
To start gdb run the following:
\begin{lstlisting}[style=BashInputStyle]
 $ gdb
\end{lstlisting}
 To load a file on start up just add the file name as a commandline argument. 
 \break
 \break
 The following prompt should be shown:
\begin{lstlisting}[style=BashInputStyle]
 (gdb)
\end{lstlisting}
\end{frame}

\begin{frame}{Interactive Shell}
        GDB is like the interactive shells you get on the command line:
\begin{itemize}
\item Similar to bash, ksh, python, and lisp
\item History of commands
\item Tab completion
\item Execution of shell commands with the \textbf{\textcolor{blue}{shell}} command
\end{itemize}
If there ever is any confusion, the built in help menu gives a description of what a command does and what arguments it takes.
\end{frame}

\begin{frame}[fragile]
\frametitle{Loading an executable}
To load an executable there are two ways. First we can add it as a command line argument to gdb:
\begin{lstlisting}[style=BashInputStyle]
$ gdb executable_name
\end{lstlisting}
If we are already running gdb we can load and unload executables at runtime:
\begin{lstlisting}[style=BashInputStyle]
(gdb) file executable_name
\end{lstlisting}
The \textbf{\textcolor{blue}{file}} command will also unload the executable without any arguments.
\end{frame}

\begin{frame}[fragile]
\frametitle{Running the program}
To run the the program:
\begin{lstlisting}[style=BashInputStyle]
(gdb) run
\end{lstlisting}
If there are no serious problems it should execute normally.
\begin{lstlisting}[style=BashInputStyle]
[inferior 1 (process 1234) exited normally]
\end{lstlisting}
If there is a problem it will return information regarding where it crashed, what function, and the function variables that caused the error.
\begin{lstlisting}[style=BashInputStyle]
Program received signal SIGSEGV, Segmentation fault.
0x0000000000400633 in main () at fib.cpp:27
27        x[c] = next;
\end{lstlisting}
\end{frame}

\begin{frame}{Debugging}
The program is running successfully without any segmentation faults. However, the result isn't what was expected.
\break
\break
Downside to using console output:
\begin{itemize}
\item Input and output is slow (cout, cin, printf, and scanf)
\item Information can scroll by too fast
\item Memory allocation
\item May not be thread safe
\item Hinders timing sensitive programs
\end{itemize}
\end{frame}
\begin{frame}{Breakpoints}
Stepping through the code can be a more efficient approach since a segmentation fault could be happening in one function but the root of the cause was a couple of function calls ago.
\break
\break
Breakpoints in gdb are similar to what is used in Visual Studio. They allow for you to stop the program from executing once it reaches a designated spot in the code.
\end{frame}

\begin{frame}[fragile]
\frametitle{Breakpoints}
To use a break just enter the command \textbf{\textcolor{blue}{break}} followed by the file and line number pair like so:
\begin{lstlisting}[style=BashInputStyle]
(gdb) break fib.cpp:27
\end{lstlisting}
This will set a break point in file fib.cpp at line 27. Anytime this line is reached the program will stop executing until another command is entered. 
\end{frame}

\begin{frame}[fragile]
\frametitle{Breakpoints}
Breakpoints can also be used to break at functions instead of line numbers. If there existed a function called sum\_array in fib.cpp the following could be done:
\begin{lstlisting}[style=BashInputStyle]
(gdb) break sum_array
\end{lstlisting}
gdb will stop executing once this function is reached.
\end{frame}

\begin{frame}{Stepping Through}
Now that the breakpoints are set we can use the \textbf{\textcolor{blue}{run}} command again to start executing the code. It will stop at these breakpoints unless an error occurs before reaching one of these points. 
\break
\break
Once a point is reached you can proceed to the next breakpoint by using the command \textbf{\textcolor{blue}{continue}}. 
\break
\break
Note: run will always restart execution from the beginning of the program.
\end{frame}

\begin{frame}{Stepping Through}
Instead of continuing on to the next breakpoint one can go to the next line of code by using the command \textbf{\textcolor{blue}{step}}. 
\break
\break
Be careful with step because it will go into the sub-routine like printf or scanf. If you don't want to go into the sub-routine use the command \textbf{\textcolor{blue}{next}} and it will treat it as a instruction.
\break
\break
Retyping the same command over and over can get tedious so gdb allows the user to just hit enter and will execute the previously entered command.
\end{frame}

\begin{frame}{Viewing Variables}
Breakpoints allow you to see if a function gets called or if the interior of a conditional statement gets executed.
\break
\break
What if a variable is the cause of the problem? 
\break
\break
GDB will allow the user to view the values of variables and registers.
\end{frame}


\begin{frame}[fragile]
\frametitle{Viewing Variables}
If you want to see all the variables in the local function that the program has stopped executing in due to a breakpoint type the command \textbf{\textcolor{blue}{info}} followed by the scope that is to be referenced. 
\break
\break
For example to view local variables type the following:
\begin{lstlisting}[style=BashInputStyle]
(gdb) info locals
\end{lstlisting}
\end{frame}

\begin{frame}{Viewing Variables}
Other variables and registers can also be viewed at any time by using the command info. Some examples include:
\begin{itemize}
\item args - Allows viewing function arguments
\item variables - Allows viewing of global variables
\item registers - Prints the names of all registers excluding floating-point and vector
\item all-registers - Prints information for all registers including floating-point and vector
\item registers regname - Prints information for a specific register
\end{itemize}
\end{frame}

\begin{frame}[fragile]
In order to view one variable the \textbf{\textcolor{blue}{print}} command followed by a variable name will print out the value of the variable. For hex use \textbf{\textcolor{blue}{print/x}} and binary use \textbf{\textcolor{blue}{print/x}}.

\begin{lstlisting}[style=BashInputStyle]
(gdb) print my_var
(gdb) print/x my_var
(gdb) print/t my_var
\end{lstlisting}
\end{frame}

\begin{frame}[fragile]
\frametitle{Viewing Variables}

How to view Memory address or information at a memory address?
\begin{lstlisting}[style=BashInputStyle]
(gdb) print array
(gdb) print &array
(gdb) print ((bigint*) 0x000d05f0).size
\end{lstlisting} 
\end{frame}

\begin{frame}{More Commands}
Useful Commands:
\begin{itemize}
\item \textbf{\textcolor{blue}{backtrace}} - summary of how your program got where it is (Done when a seg fault occurs)
\item \textbf{\textcolor{blue}{where}} - gives a function call trace of how you got to this point and shows line numbers inside files (Can be done anywhere)
\item \textbf{\textcolor{blue}{finish}} - runs until the current function has completed executing
\item \textbf{\textcolor{blue}{delete}} - deletes a specific breakpoint
\item \textbf{\textcolor{blue}{info breakpoints}} - shows information about all breakpoints
\end{itemize}
\end{frame}

\begin{frame}{More on Breakpoints}
        To create temporary breakpoints use the \textbf{\textcolor{blue}{tbreak}} command. A temporary breakpoint only stops the program once, and is then removed.
        \break
        \break
        Use the \textbf{\textcolor{blue}{disable}}/\textbf{\textcolor{blue}{enable}} command followed by the breakpoint number to turn it off/on.
\end{frame}

\begin{frame}{Conditional Breakpoints}
        Breakpoints are useful to figure out what the problem is or the general issue with a variable. 
        \break
        \break
        We don't want to keep stepping through a function if the issue only happens towards the end of an array. 
        \break
        \break
        So for this reason we would like to stop executing based on a conditional requirement.
\end{frame}

\begin{frame}[fragile]
\frametitle{Conditional Breakpoints}
Conditional breakpoints are just like normal breakpoints except for the extra conditional:
\begin{lstlisting}[style=BashInputStyle]
(gdb) break fib.cpp:27 if c >= SIZE
(gdb) break fib.cpp:27 if(c >= SIZE || c < 0)
\end{lstlisting}
We can also create a variable if we want to break when a function is called a specific amount of times.
\begin{lstlisting}[style=BashInputStyle]
(gdb) set $count = 0
(gdb) break funMethod: if ++$count == 1000
\end{lstlisting}
\end{frame}


\begin{frame}{Debugging}
GDB also has a dotfile for keeping settings between runs. .gdbinit
\begin{itemize}
\item https://github.com/cyrus-and/gdb-dashboard
\end{itemize}
When Working on a desktop you may not want to use the command line.
\begin{itemize}
\item Data Display Debugger (DDD) - Graphical Front-End for GDB
\item VS Code
\end{itemize}
\end{frame}

\begin{frame}{Useful Tools}
GDB alone will not give you all the information you need. Other tools that can work well with GDB are:
\begin{itemize}
\item Valgrind - Tools that can automatically detect many memory management and threading bugs
\item Electric Fence - Helps detect 2 common programming bugs: software that overruns the boundaries of a malloc() memory allocation, and software that touches a memory allocation that has been released by free()
\item DUMA - Similar to Electric Fence
\end{itemize}
\end{frame}

\begin{frame}{GNU PROJECT DEBUGGER}
\includegraphics{src/gdb_logo.png}
\end{frame}
\end{document}







